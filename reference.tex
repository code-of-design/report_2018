\thispagestyle{myheadings}
\markboth{参考文献}{参考文献}
\markright{}
\def\bibname{参考文献}
\begin{thebibliography}{20}

\addcontentsline{toc}{chapter}{\protect\numberline {}参考文献}

\bibitem{理学療法士}
国内渉外部: ``作業療法士の需給計画の 見直し'', 理学療法学, Vol. 18, No. 6, pp. 645-657 (1991)

\bibitem{フィットネスマシン}
竹井機器工業株式会社: ``ゲーム機能を搭載した施設向けフィットネスマシンを開発'', 報道発表資料 (2010)
\url{http://www.takei-si.co.jp/productinfo/detail/pdf/puresu.pdf}(confirmed in Feb. 2016)
\bibitem{歩行イメージ再学習}
大橋麻美, 保坂章夫, 岡田利香, 久保通宏, 関根由里, 古谷信之, 増岡泰三, 後藤博: ``脳卒中急性期片麻痺患者における歩行イメージ再学習後の歩容変化'', 理学療法学, Vol. 32,  pp. 441 (2005)

\bibitem{筑波歩行感覚提示}
田中直樹, 斉藤秀之, 飯塚陽, 矢野博明, 奥野純子, 柳久子: ``維持期脳卒中患者に対する歩行感覚提示装置を用いた歩行トレーニング効果の持続性'', 理学療法科学, Vol. 27,No.2, pp. 123--128 (2012)

\bibitem{筑波歩行感覚提示画像}
国立研究開発法人新エネルギー・産業技術総合開発機構: ``リハビリ患者がより現実に近い移動感覚で歩行機能の改善訓練ができる装置と訓練中の飽きを防ぎリハビリ効果の向上を実現する球面ディスプレイを組み合わせた新しい歩行リハビリテーションシステムの開発'', 成果事例集原稿, p. 3 (2008)

\bibitem{日立}
藤江正克, 土肥健純, 根本泰弘, 酒井昭彦, 吉田輝, 佐久間一郎, 鈴木真,``参加支援工学 バーチャルリアリティを活用した歩行訓練機器'', 日本生体医工学会, Vol. 12, No. 8, pp. 29--37 (1998) 

\bibitem{ディスプレイの違い}
小林秀明,浅井紀久夫: ``歩行動作環境において提示ディスプレイの違いが感性に与える影響'', ヒューマン情報処理, Vol. 105, pp. 143--148 (2005)

\bibitem{ロコモーション}
野間春生: ``ロコモーションとバーチャルリアリティ'', 社団法人 計測自動制御学会, Vol. 143, No. 2, pp. 133--138 (2004)

%\bibitem{足踏み}
%針山拓人, 大倉典子: ``足踏みによる歩行感覚体感デバイスの開発'', 自動制御連合講演会講演論文集, Vol.  51, pp. 223 (2008)
\bibitem{KinectV2}
KinectV2: \url{https://dev.windows.com/en-us/kinect/develop}(confirmed in Feb. 2016)

\bibitem{OculusRift}
OculusRift: \url{https://www.oculus.com/ja/rift/}(confirmed in Feb. 2016)

\bibitem{Unity}
Unity: \url{http://japan.unity3d.com/}(confirmed in Feb. 2016)

\bibitem{3D都市モデル}
Unity向け3D都市モデルデータ: \url{http://www.zenrin.co.jp/product/service/3d/asset/} (confirmed in Feb. 2016)

\bibitem{人間工学ガイド}
福田忠彦研究室: ``増補版 人間工学ガイド--感性を科学する方法--'' , サイエンティスト社, pp. 125--173 (2009)

\bibitem{average}
中村永友,山田智哉,金明哲: ``Excelで学ぶ統計•データ解析入門'', 丸善株式会社 (2011).

\bibitem{sd_zu}
和田有史,續木大介,山口拓人,木村敦,山田寛,野口薫,大山正: ``SD法を用いた視覚研究知覚属性と感情効果の研究を例として'', VISION, Vol. 15, No. 3, pp. 179--188 (2003).

\bibitem{U検定}
Wilcoxonの順位和検定(マン・ホイットニーのU検定): \url{ http://www.gen-info.osaka-u.ac.jp/MEPHAS/wilc1.html} (confirmed in Feb. 2016)

\bibitem{VR}
株式会社ポケットクエリーズ: ``ゲームのちからの VRソリューションへの応用 〜 ジオ + VR + ゲーム = ヘルスケアソリューション 〜	'', Unity Solution Conference 2014 発表資料 (2014)
\url{http://japan.unity3d.com/events/usc2014/pdf/1300_ROOM1a_PocketQueries.pdf} (confirmed in Feb. 2016)

\end{thebibliography}
\hspace{15pt}